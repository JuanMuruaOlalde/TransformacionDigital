\documentclass[11pt,a4paper]{article}
\setlength{\parindent}{0pt}
\setlength{\parskip}{0.5em}
\usepackage[utf8]{inputenc}
\usepackage[spanish]{babel}

\usepackage[a4paper,total={16cm, 24cm}]{geometry}

\usepackage{longtable}
\setlength{\tabcolsep}{12pt}

\usepackage{amsmath}
\usepackage{amsfonts}
\usepackage{amssymb}

\usepackage{graphicx}

\usepackage[colorlinks]{hyperref}
\hypersetup{colorlinks=true}
\hypersetup{urlcolor=blue}
\usepackage{cleveref}

\usepackage{authoraftertitle}
\title{¿Cómo abordar un cambio de semejante magnitud?}
\author{Juan Murua Olalde}
\date{\today}



\begin{document}

\section*{¿Cómo abordar un cambio de semejante magnitud?}
Como he venido comentando en los capítulos de {\footnotesize  \url{https://www.susosise.es/documentos/Transformacion_digital_de_la_industria.pdf}}, las grandes ventajas de la digitalización y de las nuevas tecnologías digitales residen en las nuevas formas de trabajar que posibilitan. 

Para ser realmente efectiva y desarrollar todo su potencial, la digitalización de la empresa se ha de abordar de forma holística e integral.
\\Son un montón de intervenciones, en todos los departamentos de la empresa. Todas ellas orientadas a la meta común de aportar más fluidez y transparencia a la circulación de información.

Reconozco que pueden resultar apabullantes los cambios en la cultura empresarial y en las formas de trabajar que esas intervenciones pueden requerir.

Reconozco que no se puede intentar cambiar todo de golpe. Y que plantear soluciones que asustan no es plantear soluciones, sino dar problemas mayores.

Pero las ventajas de la empresa ``4.0'' radican en el flujo fluido y transparente de la información por toda la empresa y sus aledaños. Si queremos obtenerlas, es inevitable un cambio global\ldots pero, ¿cómo abordarlo?\ldots 

\vspace{0.5cm}

La mejor solución es partir de cero: fundar una nueva empresa, basada desde su origen en las nuevas formas de trabajar. Eso son las nuevas ``startups'' que van surgiendo al calor de las nuevas tecnologias.

Esta solución, aplicada a una empresa existente ya consolidada, consiste en invertir parte de sus actuales beneficios en fundar y potenciar una nueva empresa hija totalmente independiente. Con la idea de que esta nueva empresa vaya desarrollándose (aislada totalmente de posibles ingerencias por parte de la empresa madre), creciendo dentro de las nuevas reglas de juego.

Esa empresa, apuntalada con el capital y los recursos de la empresa madre, irá captando sus propios clientes y creando su propio mercado.
\\Además, a medida que algún cliente demande servicios que la empresa madre no pueda hacer frente con su forma de trabajar ``tradicional'', puede ser transferido (discretamente) a la empresa hija.

Este proceso puede tener varios resultados finales, entre los que se cuentan estos:
\begin{itemize}
\item Si las nuevas formas de operar en el mercado no terminan de cuajar. Se ha invertido capital para obtener nuevos conocimientos. Conocimientos que se incorporarán (en la medida de lo posible) absorbiendo los activos (humanos) de la empresa hija que se liquida.
\item Si las nuevas formas de operar cuajan y son compatibles con las tradicionales. Ambas empresas prosperan. Se ha invertido capital para tener una nueva empresa en el grupo.
\item Si se diera el caso de que las nuevas formas de operar en el mercado terminan por desbancar cláramente a las formas tradicionales. Se ha invertido capital para tener una empresa que continue el legado de la empresa original; absorbiendo (en la medida de lo posible) los activos (tanto materiales como humanos) de la empresa madre que va declinando.
\end{itemize}

\vspace{0.5cm}

La solución es bastante radical y reconozco que emocionalmente es difícil de sobrellevar (\begin{footnotesize}sobre todo si se va cumpliendo algo parecido al tercero de los finales citados\end{footnotesize}). Pero, fria y racionalmente, es la solución más sencilla de llevar a la práctica.

\vspace{0.5cm}

Otra  posible solución, que es más fácil de adoptar emocionalmente, pero más difícil de implementar, es la de transformar la propia empresa desde dentro, de forma progresiva. 

Aquí el trabajo consiste en tener bien clara una arquitectura global de información bien integrada. Contando con un sistema ERP, uno PDM y uno (Big)Data, que puedan trabajar conjuntamente de forma fluida. 
\\Para irlos implementando progresivamente en los distintos departamentos y áreas de la empresa. Migrando desde los sistemas antiguos disjuntos al nuevo sistema integrado, según vayan surgiendo oportunidades para hacerlo.

Cuando surja cualquier necesidad de envergadura, aprovechar para resolverla con una intervención:
\begin{itemize}
\item Fundamentada sobre la nueva plataforma troncal.
\item Prestando especial atención a mejorar la fluidez del flujo de información.
\item Realizando los esfuerzos formativos que sean necesarios para que las personas involucradas aprendan y se habitúen a las nuevas formas de trabajar.
\end{itemize}

Es decir: ``lo que ya está, sigue como está; pero lo nuevo, se va montando sobre lo nuevo'', dando siempre prioridad a la parte nueva. Hasta que, al final, quede solo lo nuevo.

Obviamente, esto obliga a realizar puentes temporales entre partes renovadas y partes preexistentes. Pero, como temporales que son, la idea es minimizar esos puentes. Primando siempre renovar algo más de la parte preexistente si con ello se consigue reducir el alcance del puente a realizar.

Esta forma de trabajar choca frontalmente con la tendencia natural de resolver temas con el mínimo de cambios posible. De ahí que haya comentado que esta solución es más sencilla de adoptar emocionalmente, pero es más compleja de llevar a cabo adecuadamente. La tentación de mantener lo ya existente es muy fuerte.

\vspace{0.5cm}

Por último, citar la solución que parece más ``lógica y evidente''. La ``(no)solución'' de intentar realizar la transición hacia ``4.0'' suavemente, poco a poco, con pequeñas intervenciones puntuales ad-hoc. Intervenciones que van incorporando nuevas tecnologias solo cuando se ven claramente ventajosas; realizando cada intervención pensando solo en ese momento/aplicación concreto; limitándonos a lo que parece más eficiente en ese momento concreto; minimizando cambios, procurando afectar lo mínimo posible a lo ya existente;\ldots

\ldots Podremos obtener alguna de las ventajas de la digitalización y de la aplicación de nuevas tecnologias en cada intervención realizada\ldots 

\ldots Pero es muy posible que sigamos alimentando una colección de sistemas disjuntos. Sistemas ``integrados'' a base de: sincronizaciones batch nocturnas, programillas ad-hoc u hojas de cálculo para trasvasar información de un sistema a otro, o simplemente recurriendo a correos electrónicos cuando se han de realizar consultas entre departamentos. Es decir,\ldots  manteniendo silos de información que dificultan que esta fluya libremente.


\vfill

Nota: Una copia .pdf de este ensayo se puede descargar desde \url{www.susosise.es}
\\ \includegraphics[scale=0.3]{imagenes/CreativeCommons-Attribution-ShareAlike-logo}
\begin{small}\url{https://creativecommons.org/licenses/by-sa/4.0}\end{small}



\end{document}