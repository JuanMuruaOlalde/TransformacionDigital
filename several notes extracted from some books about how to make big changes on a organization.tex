\documentclass[english,14pt,a4paper,final,oneside]{article}
\setlength{\parindent}{0pt}
\setlength{\parskip}{0.5em}
\usepackage[spanish]{babel}
\usepackage[utf8]{inputenc}
\usepackage[a4paper, total={15cm, 23cm}]{geometry}

\addtolength{\skip\footins}{2pc plus 5pt}

\usepackage{enumitem}
\setlist{topsep=0pt}

\usepackage{longtable}
\setlength{\tabcolsep}{12pt}

\usepackage{amsmath}
\usepackage{amsfonts}
\usepackage{amssymb}

\usepackage{graphicx}
\graphicspath{ {./imagenes/} }

\usepackage[colorlinks]{hyperref}
\hypersetup{colorlinks=true}
\hypersetup{urlcolor=blue}
\usepackage{cleveref}

\usepackage{fancyhdr}
\fancyhf{}
%\fancyhead[RE]{\small\scshape\nouppercase{\leftmark}}
%\fancyhead[LO]{\small\scshape\nouppercase{\leftmark}}
\fancyhead[LE,RO]{\small\thepage}
\pagestyle{fancy}

\usepackage{authoraftertitle}
\title{several notes extracted from some books about \\how to make successful (big) changes on a organization}
\author{Juan Murua Olalde}
\date{05/08/2022}

\begin{document}

\begin{center}\begin{LARGE}
\MyTitle
\end{LARGE}\end{center}

\hypersetup{linkcolor=black}
%\tableofcontents

\vspace{1cm}

``Before asking someone else to do something,
\\you had to do something yourself.''

\vspace{1cm}

Not PDCA (Plan-Do-Check-Act),
\\but STPDS (See-Think-Plan-Do-See):
\begin{enumerate}
\item Collect information.
\item Analyze the information and bring issues to light.
\item Set goals and aims.
\item Draw up a winning scenario.
\item Draft a practical plan of action.
\item Carry through to the end.
\item Review and sum up (gather feedback for future reference).
\item (back to 1)
\end{enumerate}

\begin{flushright}Shigetaka Komori\end{flushright}


\vspace{1cm}


8 steps for successful (big) change processes:

\begin{enumerate}

\item Creating a sense of urgency among relevant people. (In medium size organizations, the ``relevant'' are more likely to number 100 than 5; in larger organizations 1,000 rather than 50.)

\item With urgency turned up, pull together a guiding team with the credibility, skills, connections, reputations, and formal authority required to provide change leadership.

\item The guiding team creates a sensible, clear, simple, uplifting visions and sets of strategies.

\item And communicates these visions and strategies to all people, with simple, heartfelt messages sent through many unclogged channels.
\\Here, deeds are often more important than words. Symbols speak loudly. Repetition is key.

\item A heavy dose of empowerment, to all people. Key obstacles that stop people from acting on the vision are removed.
\\The issue here is removing obstacles, not ``giving power''. You can't hand out power in a bag.

\item People are helped to produce short-term wins. The wins are critical. They provide credibility, resources, and momentum to the overal effort.
\\Without a well-managed process, careful selection of initial projects, and fast enough successes, the cynics and skeptics can sink any effort.

\item Change leaders don't let up. Momentum builds after the first wins. Early changes are consolidated. People shrewdly choose what to tackle next, then create wave after wave of change until the vision is a reality.

\item Change leaders make change stick by nurturing a new culture. A new culture --group norms of behavior and shared values-- develops through consistency of successful action over a sufficient period of time.
\\If changes float fragile on the surface. A great deal of work can be blown away by the winds of tradition in a remarkably short period of time.

\end{enumerate}

In step 1 is not urgency in some abstract sense. The core issue is the behavior of people who are ignoring how the world is changing, who are frozen in terror by the problems they see, or who do little but bitterly complain.

In step 2, the issue is the behavior of those in a position to guide change --specially regarding trust and commitment.

In step 3, the core challenge is for people to start acting in a way that will create sensible visions and strategies.

In step 4, the issue is getting sufficient people to buy into the vision via communication.

In step 5, it's acting on that communication --which for some employees will mean doing their jobs in radically new ways.

And so on throughout the process\ldots


\vspace{1.5cm}


Formal data gathering, analysis, report writing and presentations. Are the sorts of actions typically aimed at changing thinking in order to change behavior.
\\Formal analysis is important. But it has at least three major limitations:
\begin{itemize}
\item In a remarkable number of cases, you don't need it to find the big truths.
\item Analytical tools have their limitations in a turbulent world. These tools work best when parameters are know, assumptions are minimal, and the future is not fuzzy.
\item Good analysis rarely motivates people in a big way. It changes thought, but how often does it send people running out the door to act in significantly new ways?.
\end{itemize}

\vspace{0.6cm}

Compellingly showing people what the problems are and how to resolve the problems. Provoke responses that reduce feelings that slow and stifle needed change. The emotional reaction then provides the energy that propels people to push along the change process, no matter how great the difficulties.

SFC (See-Feel-Change)
\begin{itemize}
\item Find a problem. Find dramatic, eye-catching, compelling situations that help others visualize the problem or a solution to the problem. Show the need for change with a compelling objects that people can actually see, touch, and feel.
\item The visualizations awaken feelings that facilitate useful change or ease other feelings that are getting in the way of useful change.
\item The new feelings change or reinforce new behavior, sometimes very different behavior.
\end{itemize}

Successful see-feel-change tactics tend to be clever, not clumsy, and never cynically manipulative. They often have an afterglow, where the story of the event is told again and again or where there is a remaining visible sign of the event that influences additional people over time.

\vspace{0.6cm}

Obviously, you cannot have sensible change without sensible direction. So setting (general) direction must come first. Then you implement it.


\vspace{1cm}
\begin{flushright}
John P. Kotter , Dan S. Cohen
\\``The Heart of Change''
\\\begin{footnotesize}
`Real-Life Stories of How People Change Their Organizations.'
\end{footnotesize}
\\Harvard Business Review Press (2002)
\end{flushright}


\end{document}