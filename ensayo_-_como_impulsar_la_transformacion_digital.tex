\documentclass[11pt,a4paper]{article}
\setlength{\parindent}{0pt}
\setlength{\parskip}{0.5em}
\usepackage[utf8]{inputenc}
\usepackage[spanish]{babel}

\usepackage[a4paper,total={16cm, 24cm}]{geometry}

\usepackage{longtable}
\setlength{\tabcolsep}{12pt}

\usepackage{amsmath}
\usepackage{amsfonts}
\usepackage{amssymb}

\usepackage{graphicx}

\usepackage[colorlinks]{hyperref}
\hypersetup{colorlinks=true}
\hypersetup{urlcolor=blue}
\usepackage{cleveref}

\usepackage{authoraftertitle}
\title{¿Cómo impulsar la transformacion digital?}
\author{Juan Murua Olalde}
\date{16 Marzo 2021}



\begin{document}

\section*{¿Cómo impulsar la transformacion digital?}

Las tecnologias cambian. Y, con ellas, cambian las formas de trabajar. 

Cuando el ritmo de cambio es grande, la única forma de seguirlo es crear un clima de aprendizaje continuo. Fomentando una cultura empresarial donde esté bien visto intentar nuevas formas de realizar algo; donde sea posible aprender de los errores y aprovechar los aciertos.

Para favorecer este cambio de cultura empresarial podría resultar útil organizar cursos de forma regular; por lo menos una vez al año; para todas las personas de la empresa.

Cursos diferentes de los típicos de una o dos semanas, organizados para casos puntuales de adopción de nuevas herramientas.

La idea es que sean cursos:

\begin{itemize}

\item Fuera del entorno de trabajo y con dedicación exclusiva al curso. 

\item Prácticos. Cursos donde se trabajen múltiples casos de uso; aunque algunos de esos casos no tengan demasiada relación directa con el día a día del trabajo habitual.

\item Con la duración que sea necesaria; para que las personas salgan del curso sintiéndose cómodas en el manejo de las herramientas que se hayan visto en él.

\end{itemize}

La idea es acostumbrar a las personas a trabajar con nuevas herramientas. Usando esas herramientas ``tal cual salen de la caja''; usándolas siguiendo la forma en que se supone que han de funcionar; utilizarlas para realizar los trabajos para los cuales han sido diseñadas. Intentando olvidarnos durante el curso de nuestra forma de trabajar habitual, del día a día de nuestro puesto de trabajo.

La idea es salir de cada curso ``sintiéndonos cómodos'' en el manejo de la herramienta, en su aplicación a trabajos y formas de trabajar diferentes. Relacionadas con nuestro trabajo,\ldots pero diferentes.

Al principio puede haber quejas de que ``esto NO sirve para lo que hacemos nosotros''. Pero es importante que desde Dirección se dé un mensaje claro respecto a la importancia de dedicar ese tiempo a aprender cosas nuevas, aunque no estén relacionadas con el trabajo habitual del día a día de cada cual.

Realizando este tipo de cursos todos los años \footnote{Sí, estoy hablando de dedicar algo así como un mes al año, de cada persona, a formación. Al menos durante unos cuantos años, hasta que cambie el clima general de la empresa y se adopte una actitud generalizada de aprendizaje continuo}, se obtienen varios efectos beneficiosos:
\begin{itemize}
\item Llevarnos regularmente más allá de nuestra zona de confort. Acostumbrándonos a que no pasa nada por hacerlo.
\item Aprender otras formas de trabajar. Acostumbrándonos a que existen, y que pueden ser tanto o más válidas que las habituales en nuestro entorno.
\item Aumentar nuestro conocimiento de los procesos. Viendo otras partes ``aguas arriba'' o ``aguas abajo'' de la parte del proceso que realizamos habitualmente.
\item Aumentar nuestra polivalencia. Viendo que podemos abordar otros trabajos distintos a los de nuestro puesto de trabajo habitual.
\end{itemize}



\end{document}